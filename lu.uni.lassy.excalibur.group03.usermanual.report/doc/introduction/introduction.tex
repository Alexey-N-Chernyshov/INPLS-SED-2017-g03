\chapter{Introduction}
\label{chap:introduction}

\section{Scope}

Motivation for creating this User Manual is that iCrash CMS system has been
updated recently. In this new version we have to cover the main changes and
their affects on the other modules of the system. \\
This user manual should be considered as the only true. Because previous
versions doesn`t consider actual featrues of current version of iCrash system
and there is no other actual version except this neither printed nor the online
help. This document provides minimum acceptable information for knowing how to
use iCrash. \\
This User Manual is limited to the software documentation product and does not
include the processes of developing or managing software user documentation. It does not apply to specialized course materials
intended primarily for use in formal training programs. \\

\section{Purpose}

This User Manual provides minimum acceptable explanation of new
features and their interaction with previous features that are affected by new
one. This document defines:
\begin{enumerate}
  	\item new features detalized in procedures and pictures that
was added in that release and also we will describe features that was changed according to new one. 
	\item explanation of each software operation of each procedure that will be
very helpfull for other programmers that will work with that system.
	\item common errors and problems and the way of handling them, that
will be very usefull for wide audience. 
\end{enumerate} 

\section{Intended audience}

Successful user manual is the result of proper audience identification. We first
of all recommend this User Manual to administrators and coordinators of iCrash
system, in other words to these people who directly interact with the
system in their daily work. This will help for better understanding current
features, inner processes behind them and also for understanding how to resolve common errors
and problems. However, this User Manual will be very helpfull for the next command that will
work on improvement the current system. And for better understanding the entire mechanism it could be interestig for indirect users of this system for the Victim, Witness, Anonymous.
For all categories this User Manual will be helpfull for observing history of
CMS iCrash. 


\section{\mysystemname}
iCrash system is dedicated to inform about a car crash and help with crisis
handling. The purpose of this application is to take control over emergency
situations, optimize the process of their resolving, to provide mechanism of
centralized information gathering from any person involved in car crash crisis
and to guide victims during crisis handling.

\subsection{Actors \& Functionalities}

\subsubsection{Activator}
Represents a logical actor for time automatic message sending based on system’s
or environment status.\\
Functionalities:
\begin{itemize}
  \item Communication of the current time to the system
  \item Notification of the administrator that some crisis are still pending for
  a too long time
\end{itemize}

\subsubsection{Administrator}
Represents an actor responsible of administration tasks for the iCrash system.\\
Functionalities:
\begin{itemize}
  \item Adding coordinator actors to the system and its environment
  \item Deleting coordinator actors from the system and its environment
\end{itemize}

\subsubsection{Communication Company}
Represents the communication company stakeholder ensuring the input/ouput of
textual messages with humans having communication devices.\\
Functionalities:
\begin{itemize}
  \item Delivery of any SMS sent by any human to the iCrash's phone number
  \item Transmission of SMS messages from the ABC company that owns the iCrash
  system to any human having an SMS compatible device accessible using a phone
  number
\end{itemize}

\subsubsection{Coordinator}
Represents actor responsible of handling one or several crisis for the iCrash
system.\\
Functionalities:
\begin{itemize}
  \item Securely monitoring the existing alerts and crisis
  \item Securely managing alerts and crisis until their termination
\end{itemize}

\subsubsection{Messir Creator}
Represents the creator stakeholder in charge of state and environment
initialization.\\
Functionalities:
\begin{itemize}
  \item Installation of the iCrash System
  \item Definition of the values for the initial system’s state
  \item Definition of the values for the initial system’s environment
  \item Ensurance of  the integration of the iCrash system with its initial
  environment
\end{itemize}

\subsection{Operating environment}
The iCrash system is deployed across three different parts:\\
The first part is database. Database stores all the data related to crises,
alerts, system users and phone numbers of people who sent SMS message to ABC
company.\\
The second part is the server application. Server part runs the application
business logic. On one side it manipulates with the database, on the other side
clients communicate with the server in order to satisfy their objectives.\\
The third part is client side application. Client side presents only graphical
user interface for all system’s users. By manipulating with this interface users
of the system interact with the server side of the application.

\section{Document structure}
Information on how this document is organised and it is expected to be used.
Recommendations on which members of the audience should consult which sections
of the document, and explanations about the used notation (i.e. description of
formats and conventions) must also be provided.\\
This document is organized in 5 chapters.\\
The first chapter presents general information about iCrash system such as
system requirements, supported operating system, copyrights, trademark notices,
etc.\\
The second chapter contains introductory information about the system and
content of the user manual.\\
The third chapter is intended to describe user-level features of the system and
the way users interact during execution of these features. It should be used by
the users of the system in order to understand how the features of the system
work when the system is normally running. The format of these feature
descriptions is the following:
\begin{itemize}
  \item \textbf{Procedure} --- the name of a feature
  \item \textbf{Scope} --- environment of execution
  \item \textbf{Primary actor} --- an actor of the system, who initiates the
  feature
  \item \textbf{Secondary actor(s)} --- other actors who participate in feature
  execution
  \item \textbf{Goal} --- the goal which the primary actor pursues
  \item \textbf{Level} --- the level of description detalization of the feature
  \item \textbf{Main success scenario} --- the sequence of actions between an
  actor and the system
  \item \textbf{Extension} --- other alternative sequence(s) of steps during
  feature execution
\end{itemize}

The fourth chapter shows actual operations performed by the system, their
constraints, output messages and what causes these operations to execute. It can
be used to understand what happens to the system during interaction with it. The
format of operation descriptions is the following:
\begin{itemize}
  \item \textbf{Parameters} --- the input data which are used to perform an
  operation
  \item \textbf{Precondition} --- the logical condition which should be true in
  order to the operation to be performed
  \item \textbf{Post-condition} --- the logical condition which should always be
  true after the operation is performed
  \item \textbf{Output messages} --- the output information during execution
  during execution of an operation
  \item \textbf{Triggering} --- the sequence of steps which triggers system
  operation execution
\end{itemize}

The fifth chapter chapter describes the  errors which might happen during system
operation: the indications of an error, information about the reason that caused
an error and what should be done to resolve it.
