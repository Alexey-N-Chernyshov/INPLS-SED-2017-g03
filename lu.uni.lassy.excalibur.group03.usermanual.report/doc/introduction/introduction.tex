\chapter{Introduction}
\label{chap:introduction}

\section{Scope}
This section has to provide the scope of the user's manual document.
In the following some opening statements to use when providing the
information corresponding to this section. \\

Motivation for creating this User Manual is that iCrash CMS system has been
updated recently. In this new version we have to cover the main changes and
their affects on the other modules of the system. \\
This user manual should be considered as the only true. Because previous
versions doesn`t consider actual featrues of current version of iCrash system
and there is no other actual version except this neither printed nor the online
help. This document provides minimum acceptable information for knowing how to
use iCrash. \\
This User Manual is limited to the software documentation product and does not
include the processes of developing or managing software user documentation. It does not apply to specialized course materials
intended primarily for use in formal training programs. \\

This document provides \ldots
%Example: This document provides minimum acceptable information for knowing how
% to use the software system \mysystemname.


This document does not \ldots 
This document is not \ldots 
%Example: This document is not intended to provide information about how to
% connect, deploy, configure, or use any external device or
% third-party software system that is rqeuired for the correct funcitoning of
% \mysystemname.

 
This document may be used with \ldots
%This document may be used with other documents provided by third-party
% companies to have an overall view and correct understanding of the environment
% and procedures where the software system \mysystemname is aimed to be deployed
% and run.




\section{Purpose}
In this section you explain the purpose (i.e. aim, objectives) of the user's
manual. In the following some examples of opening statements to be used in this
section.

This User Manual provides minimum acceptable explanation of new
features and their interaction with previous features that are affected by new
one. This document defines:
\begin{enumerate}
  \item new features detalized in procedures and pictures that
was added in that release and also we will describe features that was changed according to new one. 
\item explanation of each software operation of each procedure that will be very helpfull for other programmers that will work with that system.
\item common errors and problems and the way of handling them, that
will be very usefull for wide audience. 
\end{enumerate} 


The purpose of this document is \ldots

This document defines \ldots

This document is meant to \ldots



\section{Intended audience}
Description of the categories of persons targeted by this document together with
the description of how they are expected to exploit the content of the document.

Successful user manual is the result of proper audience identification. We first
of all recommend this User Manual to administrators and coordinators of iCrash
system, in other words to these people who directly interact with the
system in their daily work. This will help for better understanding current
features, inner processes behind them and also for understanding how to resolve common errors
and problems. However, this User Manual will be very helpfull for the next command that will
work on improvement the current system. And for better understanding the entire mechanism it could be interestig for indirect users of this system for the Victim, Witness, Anonymous.
For all categories this User Manual will be helpfull for observing history of
CMS iCrash. 


\section{\mysystemname}
iCrash is a system dedicated to inform about car crash and help with crisis
handling.


\subsection{Actors \& Functionalities}
Overview of all the \textbf{\emph{\glspl{actor}}} interacting with the software
being them either humans (called end-users in the standard
\cite{IEEE-2001-userdocumentation}) or not. For each actor, describe the main
software functions that are offered to him. Structure of this sub-section MUST
be by actor/functionalities.


\subsection{Operating environment}
Brief overview of the infrastructure on which the software is deployed and used.

\section{Document structure}  
Information on how this document is organised and it is expected to be
used. Recommendations on which members of the audience
should consult which sections of the document, and explanations about the used
notation (i.e. description of formats and conventions) must also be provided.





