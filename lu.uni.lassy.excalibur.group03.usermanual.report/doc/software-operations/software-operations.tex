\chapter{Software operations}
\label{chap:soptware_operations}


Explain each allowed software operations (i.e. an atomic unit of treatment, a service, a functionality) including a brief description of the operation, required parameters, optional parameters, default options, required steps to trigger the operation, assumptions upon request of the operation and expected results of executing such operation.
Describe how to recognise that the operation has successfully been executed or
abnormally terminated. The template given below (i.e. section \ref{operation:MyOperation} has to be used).

Group the operations devoted to the needs of specific actors. Common
operations to several actors may be grouped and presented once to avoid redundancy.


\section{MyOperation}
\label{operation:MyOperation}
The system operator creates and adds a new crisis to the system after being
informed by a third party (citizen, organization) and selects a crisis handler for the crisis.

\begin{description}

\item \textbf{Parameters:} Reporter Personal Information, Crisis Information, Crisis Handler
\item \textbf{Precondition:} The system operator is logged in and has received information from a reporter.
\item \textbf{Post-condition:} A new crisis has been added to the system and the new crisis has been assigned to a crisis handler, the Handler has received an automatic notification from the system.
\item \textbf{Output messages:} The selected Crisis Handler will be notified
automatically once the crisis has been created.

\item \textbf{Triggering:}
\begin{enumerate}
\item From within the crisis management window fill out the required entries related to the personal information of the reporter such as name and phone number.
\item Fill out the entries related to the crisis type, impacted area, priority, description, GPS coordinates, address and finally choose a Crisis Handler from the combo box.
\item Click on the “Submit” button in and add the entry to the database.
\end{enumerate}

 
\end{description}

 
\subsection{MyExample1}
Examples should illustrate the use of \textbf{complex operations}.

Each example must show how the actor uses the software operation under
description to achieve (at least one of) its expected outcome.

It might be required to include GUI screenshots to illustrate the example.



%==========================AddCoordinator==================================
\section{RegistrationOfCoordinator}
\label{operation:RegistrationOfCoordinator}
The system Administrator adds a new Coordinator to the system. The administrator 
provides Coordinator's User ID, User name, Password and a phone number.

\begin{description}

\item \textbf{Parameters:} Coordinator User ID, Coordinator User name,
Coordinator Password, Coordinator phone number.
\item \textbf{Precondition:} The system administrator is logged in.
\item \textbf{Post-condition:} A new Coordinator has been added to the system
database.
\item \textbf{Output messages:} The coordinator was added to the system.

\item \textbf{Triggering:}
\begin{enumerate}
\item Login as Administrator.
\item From Administrator Control Panel press button Add a coordinator.
\item From Administrator Control Panel fill out the required entries related to the personal information of the Coordinator such as User ID, User name, Password and phone number.
\item Click on the button Create to add the entry to the database.
\end{enumerate}
 
\end{description}


%==========================AdministratorLogin==================================
\section{AdministratorLogin}
\label{operation:AdministratorLogin}
Administrator provides username and password to login into the system. 
Then system sends temporary confirmation code via SMS. 
Administrator fill out the code and gets access to the Administrator Control Panel.

\begin{description}

\item \textbf{Parameters:} Administrator User ID, Administrator password,
Administrator phone number.
\item \textbf{Precondition:} The system administrator is not logged in.
\item \textbf{Post-condition:} The administrator gets access to the
Administrator Control Panel.
\item \textbf{Output messages:} The administrator is logged in.

\item \textbf{Triggering:}
\begin{enumerate}
\item From Administrator Control Panel Login Window fill out the required entries as Username and Password and push on the button Continue.
\item Wait for SMS on the provided phone number with temporary confirmation code.
\item Fill out the entry confirmation code and press on the button Login to
access the Administrator Control Panel.
\end{enumerate}
 
\end{description}


%==========================CoordinatorLogin==================================
\section{CoordinatorLogin}
\label{operation:CoordinatorLogin}
Coordinator provides username and password to login into the system. 
Then system sends temporary confirmation code via SMS. 
Coordinator fill out the code and gets access to the Coordinator Control Panel.
\begin{description}

\item \textbf{Parameters:} Coordinator User ID, Coordinator password,
Coordinator phone number.
\item \textbf{Precondition:} The Coordinator is not logged in.
\item \textbf{Post-condition:} The Coordinator gets access to the
Coordinator Control Panel.
\item \textbf{Output messages:} The Coordinator is logged in.

\item \textbf{Triggering:}
\begin{enumerate}
\item From Coordinator Control Panel Login Window fill out the required entries as Username and Password and push on the button Continue.
\item Wait for SMS on the provided phone number with temporary confirmation code.
\item Fill out the entry confirmation code and press on the button Login to
access the Coordinator Control Panel.
\end{enumerate}
 
\end{description}

%==========================CoordinatorLogin==================================
\section{ConsultOfPI}
\label{operation:ConsultOfPI}
ComCompany provides all necesary information about the victim then system sends
SMS message to the Victim with all necessary infromation. 
\begin{description}

\item \textbf{Parameters:} Victim phone number, Victim location information.
\item \textbf{Precondition:} Victim selected option from list in SMS.
\item \textbf{Post-condition:} Victim received SMS message with data he or she
requested according Victim location information.
\item \textbf{Output messages:} comes right after Victim response via SMS

\item \textbf{Triggering:}
\begin{enumerate}
\item CMS informs us about alert with new Victim and Coordinator verified this
alert as valid (really happened)
\item Victim was mistaken while sending message with incorrect data or message was empty
\item Victim makes some choice and sends correct request
\end{enumerate}
 
\end{description}


