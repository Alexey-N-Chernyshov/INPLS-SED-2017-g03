\chapter{Software operations}
\label{chap:soptware_operations}

%==========================AddCoordinator==================================
\section{RegistrationOfCoordinator}
\label{operation:RegistrationOfCoordinator}
The system Administrator adds a new Coordinator to the system. The administrator 
provides Coordinator's user id, user name, password, phone number, selects
domain of expertise.

\begin{description}

\item \textbf{Parameters:} Coordinator user id, Coordinator user name,
Coordinator password, Coordinator phone number, Coordinator domain of expertise.
\item \textbf{Precondition:} The system Administrator is logged in and
Coordinaotr user id is not empty and user id is unique in the system database
and Coordinator user name is not empty and Coordinator phone number is not empty and Coordinator domain of expertise is
set.
\item \textbf{Post-condition:} A new Coordinator has been added to the system
database.
\item \textbf{Output messages:} The Administrator is notified that Coordinator
has been added to the system.

\item \textbf{Triggering:}
\begin{enumerate}
\item Login as Administrator.
\item From Administrator Control Panel press button ``Add a coordinator''.
\item From Administrator Control Panel fill out the required entries related to
the personal information of the Coordinator such as user id, user name,
password, phone number and domains of expertise.
\item Click on the button ``Create'' to add the entry to the database.
\end{enumerate}
 
\end{description}


%==========================DeleteCoordinator==================================
\section{DeletionOfCoordinator}
\label{operation:DeletionOfCoordinator}
The system Administrator deletes a Coordinator from the system by the Coordinator's user id.

\begin{description}

\item \textbf{Parameters:} Coordinator user id.
\item \textbf{Precondition:} The system Administrator is logged in and
Coordinaotr user id is not empty and the entry with Coordinator user id exists
in the system database.
\item \textbf{Post-condition:} A Coordinator with defined user id is not in the
system database.
\item \textbf{Output messages:} The Administrator is notified that Coordinator
has been deleted from the system.

\item \textbf{Triggering:}
\begin{enumerate}
\item Login as Administrator.
\item From Administrator Control Panel press button ``Delete a coordinator''.
\item From Administrator Control Panel fill out the Coordinator's user id.
\item Click on the button ``Delete'' to delete the entry frome the database.
\end{enumerate}
 
\end{description}


%==========================AdministratorLogin==================================
\section{UserLogin}
\label{operation:UserLogin}
An Administrator or Coordinator is called further User. User enters username and
password to login into the system. Then system sends temporary confirmation code
via SMS. User fill out the code and gets access to the Control Panel.

\begin{description}

\item \textbf{Parameters:} User id, password, phone number
\item \textbf{Precondition:} The User is not logged in
\item \textbf{Post-condition:} The user gets access to the Control Panel, and
username is equal to the one in database, and password is equal to the one in
database and confirmation code is equal to generated one.
\item \textbf{Output messages:} The User is notified that he is logged in.

\item \textbf{Triggering:}
\begin{enumerate}
\item From Control Panel Login Window fill out the required entries as
``username'' and ``password'' and push on the button ``Continue''.
\item Wait for SMS on the phone number had provided with temporary confirmation
code.
\item Fill out the entry ``confirmation code'' and press on the button ``Logon''
to access the Control Panel.
\end{enumerate}
 
\end{description}


%==========================CoordinatorLogin==================================
\section{CoordinatorLogin}
\label{operation:CoordinatorLogin}
Coordinator provides username and password to login into the system. 
Then system sends temporary confirmation code via SMS. 
Coordinator fill out the code and gets access to the Coordinator Control Panel.
\begin{description}

\item \textbf{Parameters:} Coordinator User ID, Coordinator password,
Coordinator phone number.
\item \textbf{Precondition:} The Coordinator is not logged in.
\item \textbf{Post-condition:} The Coordinator gets access to the
Coordinator Control Panel.
\item \textbf{Output messages:} The Coordinator is logged in.

\item \textbf{Triggering:}
\begin{enumerate}
\item From Coordinator Control Panel Login Window fill out the required entries as Username and Password and push on the button Continue.
\item Wait for SMS on the provided phone number with temporary confirmation code.
\item Fill out the entry confirmation code and press on the button Login to
access the Coordinator Control Panel.
\end{enumerate}
 
\end{description}

%==========================CoordinatorLogin==================================
\section{ConsultOfPI}
\label{operation:ConsultOfPI}
ComCompany provides all necesary information about the victim then system sends
SMS message to the Victim with all necessary infromation. 
\begin{description}

\item \textbf{Parameters:} Victim phone number, Victim location information.
\item \textbf{Precondition:} Victim selected option from list in SMS.
\item \textbf{Post-condition:} Victim received SMS message with data he or she
requested according Victim location information.
\item \textbf{Output messages:} comes right after Victim response via SMS

\item \textbf{Triggering:}
\begin{enumerate}
\item CMS informs us about alert with new Victim and Coordinator verified this
alert as valid (really happened)
\item Victim was mistaken while sending message with incorrect data or message was empty
\item Victim makes some choice and sends correct request
\end{enumerate}
 
\end{description}

%===============================CreateAlert=====================================
\section{CreateAlert}
\label{operation:CreateAlert}
Creates an instance of alert and crisis based on the SMS received from a human
(witness, victim, anonym), SMS sender's phone number, latitude and longitude of
SMS sender and SMS arrival time.

\begin{description}

\item \textbf{Parameters:} SMS message content, SMS sender's phone number, SMS
receive time, SMS sender's location, SMS sender's role in crisis (witness,
victim, anonym).
\item \textbf{Precondition:} SMS message is received.
\item \textbf{Post-condition:} Alert is created and crisis is created if there
are no crises around 100 meters.
\item \textbf{Output messages:} Crisis approving SMS message is sent to an SMS
sender via Communication Company.

\item \textbf{Triggering:}
\begin{enumerate}
\item From within the communication company management window fill out the
required entries related to the alert/crisis time, SMS sender's location, phone
number, comment and crisis attributes.
\item Click on ``Send alert'' button.
\end{enumerate}
 
\end{description}

%=================================GetAlerts=====================================
\section{GetAlerts}
\label{operation:GetAlerts}
 Gets the list of alerts depending on coordinator's domains of expertise at the
 moment of loading the system.

\begin{description}

\item \textbf{Parameters:} Alert statuses, coordinator's domains of expertise.
\item \textbf{Precondition:} Coordinator is logged in.
\item \textbf{Post-condition:} List of alerts contains all the existing alerts
except those that doesn't correspond to domains of expertise of the coordinator.
\item \textbf{Output messages:} ---

\item \textbf{Triggering:}
\begin{enumerate}
\item Coordinator logs in or coordinators switches to ``Alerts'' tab or
coordinator switches required alert status.
\end{enumerate}
 
\end{description}

%=================================GetCrises=====================================
\section{GetCrises}
\label{operation:GetCrises}
Gets the list of crises depending on coordinator’s domains of expertise at the
moment of loading the system.

\begin{description}

\item \textbf{Parameters:} Crisis status, coordinator’s domains of
expertise.
\item \textbf{Precondition:} Coordinator is logged in.
\item \textbf{Post-condition:} List of crises contains all the existing alerts
except those that doesn’t correspond to domains of expertise of the coordinator.

\item \textbf{Output messages:} ---

\item \textbf{Triggering:}
\begin{enumerate}
\item Coordinator logs in or coordinators switches to ``Crises'' tab or
coordinator switches required crisis status.
\end{enumerate}
 
\end{description}

%=================================ReportOnCrisis===============================
\section{ReportOnCrisis}
\label{operation:ReportOnCrisis}
Changes comment parameter for a crisis.

\begin{description}

\item \textbf{Parameters:} Crisis id, Report message.
\item \textbf{Precondition:} Crisis with specified crisis id exists and report
message is not empty.
\item \textbf{Post-condition:} New crisis comment is the report message.

\item \textbf{Output messages:} The crisis comment has been updated!

\item \textbf{Triggering:}
\begin{enumerate}
\item Login as coordinator.
\item Select a crisis.
\item Press ``Report of crisis'' button.
\item Fill the message field.
\item Press ``Report'' button.
\end{enumerate}
 
\end{description}

%=================================ValidateAlert===============================
\section{ValidateAlert}
\label{operation:ValidateAlert}
Changes status of an alert to ``valid''.

\begin{description}

\item \textbf{Parameters:} Alert id.
\item \textbf{Precondition:} Alert with specified alert id exists.
\item \textbf{Post-condition:} Alert with specified id has status ``valid''.

\item \textbf{Output messages:} The alert is now declared as valid!

\item \textbf{Triggering:}
\begin{enumerate}
\item Login as coordinator.
\item Select an alert.
\item Press ``Validate'' button.
\end{enumerate}
 
\end{description}

%=================================InvalidateAlert===============================
\section{ValidateAlert}
\label{operation:ValidateAlert}
Changes status of an alert to ``invalid''.

\begin{description}

\item \textbf{Parameters:} Alert id.
\item \textbf{Precondition:} Alert with specified alert id exists.
\item \textbf{Post-condition:} Alert with specified id has status ``invalid''.

\item \textbf{Output messages:} The alert is now declared as invalid!

\item \textbf{Triggering:}
\begin{enumerate}
\item Login as coordinator.
\item Select an alert.
\item Press ``Invalidate'' button.
\end{enumerate}
 
\end{description}

%=================================ChangeCrisisStatus===============================
\section{ChangeCrisisStatus}
\label{operation:ChangeCrisisStatus}
Changes status of a crisis.

\begin{description}

\item \textbf{Parameters:} Crisis id, status.
\item \textbf{Precondition:} Crisis with specified crisis id exists and status
is either pending, handled, solved, closed.
\item \textbf{Post-condition:} Crisis with specified id has status as
the status parameter.

\item \textbf{Output messages:} The crisis status has been updated!

\item \textbf{Triggering:}
\begin{enumerate}
\item Login as coordinator.
\item Select a crisis.
\item Press ``Change crisis status'' button.
\item Select status
\item Press ``Change status'' button
\end{enumerate}
 
\end{description}

%=================================InitializeDatabase=====================================
\section{InitializeDatabase}
\label{operation:InitializeDatabase}
System initializes database at the first time of system launch.

\begin{description}

\item \textbf{Parameters:} ---
\item \textbf{Precondition:} Database is not initialized.
\item \textbf{Post-condition:} All database tables are created and attributes,
domains of expertise and correlation of domains of expertise and attributes
tables are filled in.

\item \textbf{Output messages:} Database is initialized.

\item \textbf{Triggering:}
\begin{enumerate}
\item The server application is launched the first time.
\end{enumerate}
 
\end{description}