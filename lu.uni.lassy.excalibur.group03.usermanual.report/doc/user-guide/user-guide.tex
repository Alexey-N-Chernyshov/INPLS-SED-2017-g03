\chapter{Usage Guide}
\label{chap:usage_guide}

This section is aimed at describing the general use of the software, since it is
\textbf{deployed, configured} and \textbf{run}.

This software is used by actors. These actors rely on the software to perform a
set of business activities (called here procedures) aimed at reaching a
particular goal. 

These prodedures are splet in two groups:
\begin{itemize}
  \item \textbf{Multi-procedures:} which are procedures at \textbf{summary} or
  \textbf{user-goal} level involving several active or pro-active actors.
  Each of these procedures aims at illustrating intertwined
  business activities required to be performed by the involved actors
  to reach the expected goal. Each business activity between the system and an
  actor must correspond to a \textbf{system operation} instance given with actual parameter values.

  \item \textbf{Mono-procedures:} which are procedures at \textbf{summary} or
  \textbf{user-goal} level involving only one active or pro-active actor.
  Each of these procedures aims at illustrating the required business
  activities an actor has to perform to reach the expected goal. Each business
  activity between the system and the actor must correspond to a \textbf{system
  operation} instance given with actual parameter values.

\end{itemize}




Each process has to be documented using the following textual description
template \cite{armour01usecase} \textbf{BUT its content must be as low level as possible with actual values}:
\vspace{0.5cm}
\hrule
\begin{lyxlist}{PC1}
\small{
\item [\textbf{Procedure:}] ProcessMissionOne
\item [\textbf{Scope:}] Crisis Management System (\emph{CMS})
\item [\textbf{Primary Actor}:] Coordinator John
\item [\textbf{Secondary Actor(s)}:] FirstAidWorker Bob,\\
                  ExternalResourceSystem ERS
\item [\textbf{Goal:}] The intention of the Coordinator is to process mission
with ID equal to 1.
\item [\textbf{Level}:] User-goal level
\item [\textbf{Main~Success~Scenario}]:\\
1. \emph{John} instructs the \emph{CMS} to process the mission with ID equal to 12.031005\\
2. \emph{CMS} selects the internal worker \emph{Bob} to execute the mission 12.031005\\
3. \emph{CMS} instructs \emph{Bob} to behave as \emph{First Aid Worker (FAW)}\\
4. \emph{Bob} informs the \emph{CMS} of his arrival\\
5. \emph{Bob} informs the \emph{CMS} that he starts to execute the mission 12.031005\\
6. \emph{Bob} informs the \emph{CMS} that the mission 12.031005 outcome is ``Mission completed''


\item [\textbf{Extensions}]:\\
2.a None internal worker can execute the mission\\
\hspace*{0.5cm} 2.a.1 \emph{CMS} sends a request for an external resource to the \emph{ERS} actor instance\\
\hspace*{0.5cm} 2.a.2 \emph{ERS} informs \emph{CMS} that the request can be processed\\
\hspace*{0.5cm} 2.a.3 \emph{ERS} informs \emph{CMS} that \emph{Bob} can now be selected as first aid worker\\
\hspace*{0.5cm} \textbf{procedure continues at step 3}

}

\end{lyxlist}
\hrule
\vspace{0.5cm}




\Remark{Processes presentation}: processes should be introduced to the
reader in a pedagogical manner. Thus, simple and common processes should be presented before
than more complex and less utilised ones.

\Remark{Graphical User Interfaces (GUIs)}: include GUIs screenshots to show the
different stages of the process while its is performed by the actor(s).






\section{Multi-procedures}


\subsection{MyMultiProcedure1}
\ldots

\subsection{MyMultiProcedure2}
\ldots


\subsection{MyMultiProcedure3}
\ldots







\section{Mono-procedures}
Mono-procedures must be grouped by actors.


%==========================Administrator======================
\subsection{Administrator}

%-------------------------AddCoordinator----------------------
\subsubsection{AddCoordinatorAdministrator}

\vspace{0.5cm}
\hrule
\begin{lyxlist}{PC1}
\small{
\item [\textbf{Procedure:}] AddCoordinator
\item [\textbf{Scope:}] Crisis Management System (\emph{CMS})
\item [\textbf{Primary Actor}:] Administrator Bill
\item [\textbf{Secondary Actor(s)}:] Coordinator July
\item [\textbf{Goal:}] Registration of the new coordinator in the CMS.
\item [\textbf{Level}:] User-goal level
\item [\textbf{Main~Success~Scenario}]:\\
1. \emph{Bill} initializes process of a creation of the new Coordinator
(\emph{July}) in the \emph{CMS}. See figure ~\ref{fig:AdminAddCoordinator} on
page~\pageref{fig:AdminAddCoordinator}\\
2. \emph{Bill} enters July's user id, username, password and phone number to the
corresponding fields\\
3. \emph{Bill} confirms the entered data\\
4. \emph{CMS} informs Bill that the Coordinator \emph{July} is added

\item [\textbf{Extensions}]:\\
2.a Fields was not filled out\\
\hspace*{0.5cm} 2.a.1 \emph{CMS} shows the warning Incorrect data\\
\hspace*{0.5cm} \textbf{procedure continues at step 1}

}
\end{lyxlist}
\hrule
\vspace{0.5cm}


%-------------------------AuthenticationOfAdmin----------------------
\subsubsection{AuthenticationAdministrator}

\vspace{0.5cm}
\hrule
\begin{lyxlist}{PC1}
\small{
\item [\textbf{Procedure:}] AuthenticationOfAdministrator
\item [\textbf{Scope:}] Crisis Management System (\emph{CMS})
\item [\textbf{Primary Actor}:] Administrator Bill
\item [\textbf{Secondary Actor(s)}:] ComCompany
\item [\textbf{Goal:}] Authentication of Administrator in the CMS
\item [\textbf{Level}:] User-goal level
\item [\textbf{Main~Success~Scenario}]:\\
1. \emph{Bill} enters his username and password and proceeds. See figure
~\ref{fig:AdminLogin} on page~\pageref{fig:AdminLogin}\\
2. \emph{CMS} sends confirmation code via ComCompany and displays field to enter
the confirmation code \\
3. \emph{Bill} receives the SMS message with the confirmation code from
ComCompany. See figure ~\ref{fig:AdminConfirm} on
page ~\pageref{fig:AdminConfirm}\\
4. \emph{Bill} enters this code in the displayed field and proceeds\\
5. \emph{CMS} displays to \emph{Bill} Administrator control panel\\

\item [\textbf{Extensions}]:\\
2.a Fields was not filled out\\
\hspace*{0.5cm} 2.a.1 \emph{CMS} shows the warning Incorrect data: field is
empty\\
\hspace*{0.5cm} \textbf{procedure continues at step 1}

2.b Entered data is invalid\\
\hspace*{0.5cm} 2.a.1 \emph{CMS} shows the warning Incorrect data\\
\hspace*{0.5cm} \textbf{procedure continues at step 1}

2.c Confirmation code is incorrect\\
\hspace*{0.5cm} 2.a.1 \emph{CMS} shows the warning Wrong identification information\\
\hspace*{0.5cm} \textbf{procedure continues at step 1}

}
\end{lyxlist}
\hrule
\vspace{0.5cm}


%==========================Coordinator======================
\subsection{Coordinator}

\subsubsection{AuthenticationCoordinator}

\vspace{0.5cm}
\hrule
\begin{lyxlist}{PC1}
\small{
\item [\textbf{Procedure:}] AuthenticationOfCoordinator
\item [\textbf{Scope:}] Crisis Management System (\emph{CMS})
\item [\textbf{Primary Actor}:] Coordinator Gonzalez
\item [\textbf{Secondary Actor(s)}:] ComCompany
\item [\textbf{Goal:}] Authentication of Coordinator in the CMS
\item [\textbf{Level}:] User-goal level
\item [\textbf{Main~Success~Scenario}]:\\
1. \emph{Gonzalez} enters his username and password and proceeds. See figure
~\ref{fig:CoordinatorLogin} on page~\pageref{fig:CoordinatorLogin}\\
2. \emph{CMS} sends confirmation code via ComCompany and displays field to enter
the confirmation code\\
3. \emph{Gonzalez} receives the SMS message with the confirmation code from
ComCompany. See figure ~\ref{fig:CoordinatorConfirm} on
page ~\pageref{fig:CoordinatorConfirm}\\
4. \emph{Gonzalez} enters this code in the displayed field and
proceeds\\
5. \emph{CMS} displays to \emph{Gonzalez} Administrator control panel

\item [\textbf{Extensions}]:\\
2.a Fields was not filled out\\
\hspace*{0.5cm} 2.a.1 \emph{CMS} shows the warning Incorrect data: field is
empty\\
\hspace*{0.5cm} \textbf{procedure continues at step 1}

2.b Entered data is invalid\\
\hspace*{0.5cm} 2.a.1 \emph{CMS} shows the warning Incorrect data\\
\hspace*{0.5cm} \textbf{procedure continues at step 1}

2.c Confirmation code is incorrect\\
\hspace*{0.5cm} 2.a.1 \emph{CMS} shows the warning Wrong identification information\\
\hspace*{0.5cm} \textbf{procedure continues at step 1}

}
\end{lyxlist}
\hrule
\vspace{0.5cm}



\subsection{My-Actor2}

\subsubsection{MyProcedure1MyActor2}
\ldots

\subsubsection{MyProcedure2MyActor2}
\ldots














