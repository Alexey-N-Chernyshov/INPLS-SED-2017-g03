\chapter{Usage Guide}
\label{chap:usage_guide}

This section is aimed at describing the general use of the software, since it is
\textbf{deployed, configured} and \textbf{run}.

This software is used by actors. These actors rely on the software to perform a
set of business activities (called here procedures) aimed at reaching a
particular goal. 

These prodedures are splet in two groups:
\begin{itemize}
  \item \textbf{Multi-procedures:} which are procedures at \textbf{summary} or
  \textbf{user-goal} level involving several active or pro-active actors.
  Each of these procedures aims at illustrating intertwined
  business activities required to be performed by the involved actors
  to reach the expected goal. Each business activity between the system and an
  actor must correspond to a \textbf{system operation} instance given with actual parameter values.

  \item \textbf{Mono-procedures:} which are procedures at \textbf{summary} or
  \textbf{user-goal} level involving only one active or pro-active actor.
  Each of these procedures aims at illustrating the required business
  activities an actor has to perform to reach the expected goal. Each business
  activity between the system and the actor must correspond to a \textbf{system
  operation} instance given with actual parameter values.

\end{itemize}




Each process has to be documented using the following textual description
template \cite{armour01usecase} \textbf{BUT its content must be as low level as possible with actual values}:
\vspace{0.5cm}
\hrule
\begin{lyxlist}{PC1}
\small{
\item [\textbf{Procedure:}] ProcessMissionOne
\item [\textbf{Scope:}] Crisis Management System (\emph{CMS})
\item [\textbf{Primary Actor}:] Coordinator John
\item [\textbf{Secondary Actor(s)}:] FirstAidWorker Bob,\\
                  ExternalResourceSystem ERS
\item [\textbf{Goal:}] The intention of the Coordinator is to process mission
with ID equal to 1.
\item [\textbf{Level}:] User-goal level
\item [\textbf{Main~Success~Scenario}]:\\
1. \emph{John} instructs the \emph{CMS} to process the mission with ID equal to 12.031005\\
2. \emph{CMS} selects the internal worker \emph{Bob} to execute the mission 12.031005\\
3. \emph{CMS} instructs \emph{Bob} to behave as \emph{First Aid Worker (FAW)}\\
4. \emph{Bob} informs the \emph{CMS} of his arrival\\
5. \emph{Bob} informs the \emph{CMS} that he starts to execute the mission 12.031005\\
6. \emph{Bob} informs the \emph{CMS} that the mission 12.031005 outcome is ``Mission completed''


\item [\textbf{Extensions}]:\\
2.a None internal worker can execute the mission\\
\hspace*{0.5cm} 2.a.1 \emph{CMS} sends a request for an external resource to the \emph{ERS} actor instance\\
\hspace*{0.5cm} 2.a.2 \emph{ERS} informs \emph{CMS} that the request can be processed\\
\hspace*{0.5cm} 2.a.3 \emph{ERS} informs \emph{CMS} that \emph{Bob} can now be selected as first aid worker\\
\hspace*{0.5cm} \textbf{procedure continues at step 3}

}
\end{lyxlist}
\hrule
\vspace{0.5cm}




\Remark{Processes presentation}: processes should be introduced to the
reader in a pedagogical manner. Thus, simple and common processes should be presented before
than more complex and less utilised ones.

\Remark{Graphical User Interfaces (GUIs)}: include GUIs screenshots to show the
different stages of the process while its is performed by the actor(s).


\section{Multi-procedures}

\subsection{DoAlert}
\vspace{0.5cm}
\hrule
\begin{lyxlist}{PC1}
\small{
\item [\textbf{Procedure:}] DoAlert
\item [\textbf{Scope:}] Crisis Management System (\emph{CMS})
\item [\textbf{Primary Actor}:] Communication Company Vodafone
\item [\textbf{Secondary Actor(s)}:] Witness Gerald, Coordinator Natalie, Coordinator Bob
\item [\textbf{Goal:}] To inform ABC company about Gerald's crisis
\item [\textbf{Level}:] User-goal level

\item [\textbf{Main~Success~Scenario}]:\\
1. \emph{Gerald} uses his cell phone to send SMS message ``I see the car crash
with fire'' to Vodafone communication company\\
2. \emph{Vodafone} sees Communication Company Control Panel\\
3. \emph{Vodafone} sets type of a person as witness, enters Gerald's phone
number, date and time, latitude, longitude and comment ``very important
accident''\\
4. \emph{Vodafone} chooses attribute ``fire''\\
5. \emph{Gerald's} receives SMS message ``Your alert has been registered. We
will handle it and keep you informed.''\\
6. Alert and corresponding crisis are shown in \emph{Natalie's} control panel,
but not shown in \emph{Bob’s} coordinator panel (because \emph{Bob} is not an
expert in required domain).

\item [\textbf{Extensions}]:\\
3.a Fields was not filled out\\
\hspace*{0.5cm} 3.a.1 \emph{CMS} shows the warning Incorrect data\\
\hspace*{0.5cm} \textbf{procedure continues at step 3}\\
4.a No attribute is chosen\\
\hspace*{0.5cm} 3.a.1 \emph{CMS}shows the warning Incorrect data 
(at least one attribute should be chosen)\\
\hspace*{0.5cm} \textbf{procedure continues at step 4}
}
\end{lyxlist}
\hrule
\vspace{0.5cm}


\section{Mono-procedures}
Mono-procedures must be grouped by actors.

%==========================ComCompany======================
\subsection{ComCompany}

\vspace{0.5cm}
%\section{PI}
\subsubsection{ConsultOfPI}
\textbf{Procedure:} ConsultOfPI \\
\textbf{Scope:} Crisis Management System (CMS) \\
\textbf{Primary Actor:} ComCompany Vodafone, Victim Diana \\
\textbf{Secondary Actor(s):} CMS \\
\textbf{Goal:} Consult ComCompany a list of PI classified by categories. \\
\textbf{Level:} User-goal level \\
\textbf{Main success Scenario:} 
\begin{enumerate}
	\item CMS sends to victim Diana SMS with list of PI classified by categories
	via ComCompany Vodafone see the example ~\ref{fig:Selectcategory} on
	page~\pageref{fig:Selectcategory}\\
	\item Victim Diana selects from list of options which are presented in SMS and
	send his choice in response message via ComCompany Vodafone example on
	~\ref{fig:Victimchoice} on page~\pageref{fig:Victimchoice}\\
	\item CMS sends SMS with requested information (including a map and the route)
	via ComCompany Vodafone like in example ~\ref{fig:Answeronselection} on
	page~\pageref{fig:Answeronselection}\\
	\item Victim Diana took all the necessary information or she sends another SMS
	with a new choice(procedure starts from 2 point) Extension:
\end{enumerate}	
\textbf{Extension:} \\ 
\textbf{2.1} In case Victim sends empty message or with options that is
not provided in list of options � Victim receive message with text �Incorrect data. Please retry �(Here content of a previous message with list of options) � �

%==========================Administrator======================
\subsection{Administrator}

%-------------------------AddCoordinator----------------------
\subsubsection{AddCoordinatorAdministrator}

\vspace{0.5cm}
\hrule
\begin{lyxlist}{PC1}
\small{
\item [\textbf{Procedure:}] AddCoordinator
\item [\textbf{Scope:}] Crisis Management System (\emph{CMS})
\item [\textbf{Primary Actor}:] Administrator Bill
\item [\textbf{Secondary Actor(s)}:] Coordinator July
\item [\textbf{Goal:}] Registration of the new coordinator July in the CMS.
\item [\textbf{Level}:] User-goal level
\item [\textbf{Main~Success~Scenario}]:\\
1. \emph{Bill} initializes process of a creation of the new Coordinator
(\emph{July}) in the \emph{CMS} by pushing the button ``Add a coordinator''. See
figure ~\ref{fig:AdminAddCoordinator} on page~\pageref{fig:AdminAddCoordinator}\\
2. \emph{Bill} enters July's user id, username, password and phone number to the
corresponding fields. \emph{Bill} chooses \emph{July}'s domains of expertise
from the list all domains of expertise\\
4. \emph{Bill} confirms the entered data by clickinng button ``Create''\\
5. \emph{CMS} informs \emph{Bill} that the Coordinator \emph{July} is added

\item [\textbf{Extensions}]:\\
2.a Fields user id, or user name, or password were not filled out\\
\hspace*{0.5cm} 2.a.1 \emph{CMS} shows the warning Incorrect data\\
\hspace*{0.5cm} \textbf{procedure continues at step 1}

}
\end{lyxlist}
\hrule
\vspace{0.5cm}


%-------------------------AuthenticateAdmin----------------------
\subsubsection{AuthenticateAdministrator}

\vspace{0.5cm}
\hrule
\begin{lyxlist}{PC1}
\small{
\item [\textbf{Procedure:}] AuthenticateAdministrator
\item [\textbf{Scope:}] Crisis Management System (\emph{CMS})
\item [\textbf{Primary Actor}:] Administrator Bill
\item [\textbf{Secondary Actor(s)}:] ComCompany
\item [\textbf{Goal:}] Authentication of Administrator in the CMS
\item [\textbf{Level}:] User-goal level
\item [\textbf{Main~Success~Scenario}]:\\
1. \emph{Bill} enters his username and password and proceeds by clicking
``Logon'' button
See figure ~\ref{fig:AdminLogin} on page~\pageref{fig:AdminLogin}\\
2. \emph{CMS} sends confirmation code via ComCompany and displays field to enter
the confirmation code \\
3. \emph{Bill} receives the SMS message with the confirmation code from
\emph{ComCompany} and enters this code in the displayed field ``Confirmation
code'' and proceedes by clicking button ``Logon''. See figure ~\ref{fig:AdminConfirm}
on page ~\pageref{fig:AdminConfirm}\\
4. Administrator control panel is displayed to \emph{Bill} 

\item [\textbf{Extensions}]:\\
2.a Fields user id, or username, or password was not filled out\\
\hspace*{0.5cm} 2.a.1 \emph{CMS} shows the warning Incorrect data: field is
empty\\
\hspace*{0.5cm} \textbf{procedure continues at step 1}

2.b Entered uesername or password is invalid\\
\hspace*{0.5cm} 2.a.1 \emph{CMS} shows the warning Incorrect data\\
\hspace*{0.5cm} \textbf{procedure continues at step 1}

2.c Confirmation code is incorrect\\
\hspace*{0.5cm} 2.a.1 \emph{CMS} shows the warning Wrong identification information\\
\hspace*{0.5cm} \textbf{procedure continues at step 1}

}
\end{lyxlist}
\hrule
\vspace{0.5cm}


%==========================Coordinator======================
\subsection{Coordinator}

\subsubsection{AuthenticateCoordinator}

\vspace{0.5cm}
\hrule
\begin{lyxlist}{PC1}
\small{
\item [\textbf{Procedure:}] AuthenticateCoordinator
\item [\textbf{Scope:}] Crisis Management System (\emph{CMS})
\item [\textbf{Primary Actor}:] Coordinator Gonzalez
\item [\textbf{Secondary Actor(s)}:] ComCompany
\item [\textbf{Goal:}] Authentication of Coordinator in the CMS
\item [\textbf{Level}:] User-goal level
\item [\textbf{Main~Success~Scenario}]:\\
1. \emph{Gonzalez} enters his username and password and proceeds by clicking
``Logon button''. See figure ~\ref{fig:CoordinatorLogin} on
page~\pageref{fig:CoordinatorLogin}\\
2. \emph{CMS} sends confirmation code via ComCompany and displays field to enter
the confirmation code\\
3. \emph{Gonzalez} receives the SMS message with the confirmation code from
ComCompany and enters this code in the displayed field ``Confirmation code''
and proceeds by clicling button ``Logon''. See figure
~\ref{fig:CoordinatorConfirm} on page ~\pageref{fig:CoordinatorConfirm}\\
4. \emph{CMS} displays to \emph{Gonzalez} Administrator control panel

\item [\textbf{Extensions}]:\\
2.a Fields user id, or user name, or password was not filled out\\
\hspace*{0.5cm} 2.a.1 \emph{CMS} shows the warning Incorrect data: field is
empty\\
\hspace*{0.5cm} \textbf{procedure continues at step 1}

2.b Entered username or password is invalid\\
\hspace*{0.5cm} 2.a.1 \emph{CMS} shows the warning Incorrect data\\
\hspace*{0.5cm} \textbf{procedure continues at step 1}

2.c Confirmation code is incorrect\\
\hspace*{0.5cm} 2.a.1 \emph{CMS} shows the warning Wrong identification information\\
\hspace*{0.5cm} \textbf{procedure continues at step 1}

}
\end{lyxlist}
\hrule
\vspace{0.5cm}


