% Last Modification:
% @author AUTHOR_NAME
% @date TODAY_DATE

\chapter{Introduction}
\label{chap:introduction}

\section{Overview}
Motivation for creating this Specification report is that iCrash CMS system has
been updated recently. In this new version we have to cover the main changes and
their affects on the other modules of the system.\\
This Specification report should be considered as the only true. Because
previous versions doesn`t consider actual features of current version of iCrash
system and there is no other actual version except this neither printed nor the
online help. This document provides minimum acceptable information for knowing
how to use iCrash.\\
This Specification report is limited to the software documentation product and
does not include the processes of developing or managing software user
documentation. It does not apply to specialized course materials intended
primarily for use in formal training programs.\\

\section{Purpose and recipients of the document}
This document is an analysis document complying with the Messir methodology. Its
intent is to provide an example of a precise specification of the functional
properties of the iCrash system. The recipients of this document are:
\begin{itemize}
  \item the iCrash system buyes company (ABC): this document is used as a
  contractual document jointly with any other document considered as useful in
  order to have a higher degree of precision in requirement description. It is
  also used as a basis document for the iCrash system validation using
  specification based testing.
  \item the iCrash system development company (ADC) is expected to use this
  document as the basis for development (mainly design, implementation,
  maintenance). It is also used for verification and validation using test plans
  defined using the analysis models described in this document and according to
  the Messir methodology.
\end{itemize}
 
\section{Application Domain}
iCrash system is dedicated to inform about a car crash and help with crisis
handling. The purpose of this application is to take control over emergency
situations, optimize the process of their resolving, to provide mechanism of
centralized information gathering from any person involved in car crash crisis
and to guide victims during crisis handling.
 
\section{Definitions, acronyms and abbreviations}

\subsection{Activator}
Represents a logical actor for time automatic message sending based on system’s
or environment status.\\
Functionalities:
\begin{itemize}
  \item Communication of the current time to the system
  \item Notification of the administrator that some crisis are still pending for
  a too long time
\end{itemize}

\subsection{Administrator}
Represents an actor responsible of administration tasks for the iCrash system.\\
Functionalities:
\begin{itemize}
  \item Adding coordinator actors to the system and its environment
  \item Deleting coordinator actors from the system and its environment
\end{itemize}

\subsection{Communication Company}
Represents the communication company stakeholder ensuring the input/ouput of
textual messages with humans having communication devices.\\
Functionalities:
\begin{itemize}
  \item Delivery of any SMS sent by any human to the iCrash's phone number
  \item Transmission of SMS messages from the ABC company that owns the iCrash
  system to any human having an SMS compatible device accessible using a phone
  number
\end{itemize}

\subsection{Coordinator}
Represents actor responsible of handling one or several crisis for the iCrash
system.\\
Functionalities:
\begin{itemize}
  \item Securely monitoring the existing alerts and crisis
  \item Securely managing alerts and crisis until their termination
\end{itemize}

\subsection{Messir Creator}
Represents the creator stakeholder in charge of state and environment
initialization.\\
Functionalities:
\begin{itemize}
  \item Installation of the iCrash System
  \item Definition of the values for the initial system’s state
  \item Definition of the values for the initial system’s environment
  \item Ensurance of  the integration of the iCrash system with its initial
  environment
\end{itemize}

\section{Document structure}
The document structure is designed to be coherent with the Messir methodology.
Section 2. provides a general description of the system purpose, its users, its
environment and some general non functional requirements. The system operation
triggered by events sent by the external actors belonging to the environment are
described in Section 3. The iCrash concepts used to represent the any persistent
or transient information is given in Section 4. The precise specification of the
system operations in term of system's state changes, events sent together with
the constraints on the allowed sequences of system operations are described in
Section 5.
