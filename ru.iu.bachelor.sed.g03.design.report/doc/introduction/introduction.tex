\chapter{Introduction}
\label{chap:introduction}


\section{Overview}
This software system is aimed at \ldots




\section{Purpose and recipients of the document}
This document is a design document. The aim of this document is to provide an
example of how the design of a particular software system should be documented. 

The recipient of this document is the development company (ADC) in charge
of delivering the software system. The company's developers are
expected to use this document as the basis for carrying out the actual
development and deployment of the product (i.e. implementation, testing
and maintenance).



\section{Definitions, acronyms and abbreviations}

\subsubsection{Activator}
Represents a logical actor for time automatic message sending based on system’s
or environment status.\\
Functionalities:
\begin{itemize}
  \item Communication of the current time to the system
  \item Notification of the administrator that some crisis are still pending for
  a too long time
\end{itemize}

\subsubsection{Administrator}
Represents an actor responsible of administration tasks for the iCrash system.\\
Functionalities:
\begin{itemize}
  \item Adding coordinator actors to the system and its environment
  \item Deleting coordinator actors from the system and its environment
\end{itemize}

\subsubsection{Communication Company}
Represents the communication company stakeholder ensuring the input/ouput of
textual messages with humans having communication devices.\\
Functionalities:
\begin{itemize}
  \item Delivery of any SMS sent by any human to the iCrash's phone number
  \item Transmission of SMS messages from the ABC company that owns the iCrash
  system to any human having an SMS compatible device accessible using a phone
  number
\end{itemize}

\subsubsection{Coordinator}
Represents actor responsible of handling one or several crisis for the iCrash
system.\\
Functionalities:
\begin{itemize}
  \item Securely monitoring the existing alerts and crisis
  \item Securely managing alerts and crisis until their termination
\end{itemize}

\subsubsection{Messir Creator}
Represents the creator stakeholder in charge of state and environment
initialization.\\
Functionalities:
\begin{itemize}
  \item Installation of the iCrash System
  \item Definition of the values for the initial system’s state
  \item Definition of the values for the initial system’s environment
  \item Ensurance of  the integration of the iCrash system with its initial
  environment
\end{itemize}

  
\section{Document structure} 
This document is organised as follows: Section \ref{chap:AM} provides a general
overview of the main concepts gathered during the analysis phase, in particular those concerning the software system abstract
types, as well as the actors that interact with the
software system through their interfaces. 

The technologies used not only during the design and development phase, but 
also those required to make the software system runnable are presented in
Section~\ref{chap:techFrm}.

The architecture of the software system to be implemented and deployed is
described in Section \ref{chap:arch}. This section presents the components of
the software system architecture along with their interactions, both from the static and dynamic viewpoints.

The detailed design of each \gls{system operation} is given in Section
\ref{chap:detDesign}, whereas Section \ref{chap:know_limitations} enumerates the
current limitations of the software system at the writing time of this document.

Next, Section \ref{chap:testing} presents the different test cases used to
verify the correct behaviour of the software system's functional and not functional requirements.

Finally, Section \ref{chap:final_conclusion} draws the
conclusion achieved during the design and implementation of the software system.
 
 