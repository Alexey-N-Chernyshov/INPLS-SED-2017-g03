\chapter{Technological frameworks}
\label{chap:techFrm}


All technological frameworks used during the development phase of your product
are represented in this section.

\section{Java Programming Language and Environment}
\label{sec:Java}
\textbf{Java} is general-purpose multi-paradigm, class-based language with high
portability of the software. Software writen in \textbf{Java} can be executed on
any computational device with compatability to execute \textbf{Java Bytecode}
(hardware or software compatability possible). In addition, \textbf{Java Runtime
Environment} (JRE) provides wide range of tools and libraries for most common
tasks. \textbf{Java} provides a technology called \textbf{Remote Method
Invocation} (RMI) that helps to connect distributed nodes and bind computations
on them.

\section{Eclipse}
\label{sec:eclipse}
\textbf{Eclipse} is an integrated development environment (IDE) used in computer
programming, and is the most widely used Java IDE. It contains a base workspace
and an extensible plug-in system for customizing the environment. Eclipse is
written mostly in Java and its primary use is for developing Java applications,
but it may also be used to develop applications in other programming languages.

\section{MySQL}
\label{sec:MySQL}
\textbf{MySQL} is a widely-used open-source relational database
management system (RDBMS). \textbf{MySQL} is used as a fast, reliable and
distributed database storage system with support of \textbf{SQL}. Many of the
world's largest and fastest-growing organizations including Facebook, Google,
Adobe, Alcatel Lucent and Zappos rely on MySQL to save time and money powering 
their high-volume Web sites, business-critical systems and packaged software.

\section{Vagrant}
\label{sec:vagrant}
\textbf{Vagrant} is an open-source software product for building and maintaining
portable virtual development environments based on \textbf{VirtualBox}
virtual machines. \textbf{iCrash} uses \textbf{Vagrant} to deploy working
environment, such as server application and database instances.

\section{Git}
\label{sec:Git}
\textbf{Git} is a free and open source distributed version control system designed to handle 
everything from small to very large projects with speed and efficiency.
\textbf{Git} is easy to learn and has a tiny footprint with lightning fast
performance. It outclasses SCM tools like Subversion, CVS, Perforce, and ClearCase 
with features like cheap local branching, convenient staging areas, and multiple workflows.

\section{draw.io}
\label{sec:drawio}
\textbf{Draw.io} is a free app on Google drive to create a chart that allows you
to draw flowcharts, UML and diagrams.

\section{Pencil}
\label{sec:pencil}
\textbf{Pencil} is built for the purpose of providing a free and
open-source GUI prototyping tool that people can easily install and use to
create mockups in popular desktop platforms. \textbf{Pencil} provides various
built-in shapes collection for drawing different types of user interface ranging
from desktop to mobile platforms.

\section{Excalibur}
\label{sec:excalibur}
\textbf{Excalibur} is a tool supporting the \textbf{Messir} methodology, a
Scientific Method for the Software Engineering Master, used in Software
Engineering Lectures at bachelor and master levels. Excalibur tool covers the
phase of Requirements Analysis and its main features are requirements analysis
specification (its own DSL), requirements report generation (latex/pdf) and
requirements simulation (prolog). It relies on Eclipse technologies as XText for
textual specification and Sirius for graphical views of the textual
specifications.

\section{LaTeX}
\label{sec:latex}
\textbf{LaTeX} is a document preparation system that uses markup tagging
conventions to define the general structure of a document (such as article,
book, and letter), to stylise text throughout a document (such as bold and
italics), and to add citations and cross-references. A TeX distribution such as
TeX Live or MikTeX is used to produce an output file (such as PDF or DVI)
suitable for printing or digital distribution. Within the typesetting system,
its name is stylised as LATEX.
