\chapter{Introduction}
\label{chap:introduction}


\section{Overview}
This software system is aimed at \ldots




\section{Purpose and recipients of the document}
This document is a design document. The aim of this document is to provide an
example of how the design of a particular software system should be documented. 

The recipient of this document is the development company (ADC) in charge
of delivering the software system. The company's developers are
expected to use this document as the basis for carrying out the actual
development and deployment of the product (i.e. implementation, testing
and maintenance).





\section{Definitions, acronyms and abbreviations}
TODO


  
\section{Document structure} 
This document is organised as follows: Section \ref{chap:AM} provides a general
overview of the main concepts gathered during the analysis phase, in particular those concerning the software system abstract
types, as well as the actors that interact with the
software system through their interfaces. 

The technologies used not only during the design and development phase, but 
also those required to make the software system runnable are presented in
Section~\ref{chap:techFrm}.

The architecture of the software system to be implemented and deployed is
described in Section \ref{chap:arch}. This section presents the components of
the software system architecture along with their interactions, both from the static and dynamic viewpoints.

The detailed design of each \gls{system operation} is given in Section
\ref{chap:detDesign}, whereas Section \ref{chap:know_limitations} enumerates the
current limitations of the software system at the writing time of this document.

Next, Section \ref{chap:testing} presents the different test cases used to
verify the correct behaviour of the software system's functional and not functional requirements.

Finally, Section \ref{chap:final_conclusion} draws the
conclusion achieved during the design and implementation of the software system.
 
 